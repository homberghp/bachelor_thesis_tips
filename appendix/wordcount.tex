\chapter{Counting words}

It is mandatory to include the word count of your thesis in the
information page. There are many constructs shown on the Internet but
we choose to go for the simplest solution possible.

\section{What to count}
It is resonable to only count the words in the report proper.
This implies only counting the words in the chapters, which is easy
enough by using the \texttt{texcount} command with a \define{\gls{glob}} for the
chapters: \texttt{chapters/*.tex}

\section{How to count}
\begin{lstlisting}[language=sh]
texcount  -inc -sum  -1 chapters/*.tex > wordcount.txt
\end{lstlisting}

\section{When to count}
To make sure that the word count in the file is updated in the
whenever a chapter file changes, we use the Makefile.

In the Makefile we make the \texttt{wordcount.txt} file dependent on the
chapter files with this simple rule:

\begin{lstlisting}[language=make,showtabs]
wordcount.txt: chapters/*.tex
	texcount  -inc -sum  -1 chapters/*.tex > wordcount.txt
\end{lstlisting}

and make the main.pdf file dependent on the \texttt{wordcount.txt} file as
well.

When that does not work for you, maybe, because you cannot or will not
use \texttt{make} in your environment, then execute the
\texttt{texcount} as given above in the terminal.

Even simpler, delete the \texttt{wordcount.txt} file and run pdflatex
again.

