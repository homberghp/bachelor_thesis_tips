\renewcommand\TheFile{ch08_citingsimple.tex}

\begin{savequote}[15cm]
  \raggedleft
\sffamily
Everything should be made as simple as possible, but not simpler.
\qauthor{Albert Einstein}
\end{savequote}
\chapter{Citing is simple}

Adding proper references and citations to you document can be a real
burden.
Not so in \LaTeX\ where you can use the facilities of a kind ``database''
and many entries available on the web that be added to that database.

This database can be one file, but just as well a set of files, which
you use to organize you bibliography. The files with these data are
called .bib files. 

\section{bibtex}
Using bibtex is easy.
For instance if your bib contains this entry \cite{bibtexsite}
\begin{lstlisting}[language=BibTeX]
@misc{ Nobody06,
       author = "Nobody Jr",
       title = "My Article",
       year = "2006" }
\end{lstlisting}
\lstset{language=BibTeX}
Then citing Nobody~\cite{Nobody06} is easy as
pie:\lstinline|\cite{Nobody06}|

Then you need to add one compilation step to your normal workflow:

\begin{lstlisting}[language=sh]
$ latex myarticle
$ bibtex myarticle
$ latex myarticle
$ latex myarticle
\end{lstlisting}

You can of course easily add that to the makefile introduced in an
earlier chapter.

Bibtex is the standard you want to use if preparing an article of a
journal or magazine. The journal typically also prescribes a specific
bibliography style (defined in a .bst file), which is most likely
already define for or by that journal

\subsection{Biblatex and biber}
A bit more modern is biblatex, which has a much simpler definition
format for bst files. The is a separate \textbf{biber} program to do
the processing instead of the bibtex run.
I have used biber in this version of this latex sample.
