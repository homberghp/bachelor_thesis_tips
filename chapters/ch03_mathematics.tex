\renewcommand\TheFile{ch03_mathematics.tex}
% Line numbering does not work well with diplay math? 
% that is why I use the following macro
% to end a paragraph just before a displaymath
% without getting to much vertical space
\newcommand\premathpar{\vspace{-2\parskip}\par}
% To have linenumbering, add a \par before each displaymath
% The text below was taken from the sample.tex document by 
% \author{Matthias K. Gobbert}
\begin{savequote}[15cm]
  \vspace{-30mm}
  \raggedleft
  \sffamily
  Beautiful math is the purpose of \TeX\ and \LaTeX.
  \qauthor{Mantra of \TeX-ies}
\end{savequote}
\chapter{Mathematics to show off}
% first som definitions
% some personal command definitions as examples:
\newcommand{\half}{\frac{1}{2}}
\newcommand{\eps}{\varepsilon}
\newcommand{\rh}{\rho}
\newcommand{\mtheta}{\vartheta}
\newcommand{\ph}{\varphi}
% this command is for partial derivatives and takes 2 input arguments:
\newcommand{\der}[2]{\frac{\partial {#1}}{\partial {#2}}}

Here, you see how a mathematical equation can be generated in line, for
instance $f(x) = \frac{1}{1+25 x^2}$.
The \verb+$+-symbols enclose the formula.
As a so-called displayed formula, it would look like\premathpar
\begin{displaymath}
  f(x) = \frac{1}{1+25 x^2}.
\end{displaymath}
It is customary that mathematical functions are \emph{not} set in math-italics,
so \LaTeX\ has the basic ones pre-defined; you should use the commands
\verb+\cos+, \verb+\exp+, etc.\ to get $f_1(x) = \cos x$,
$f_2(x) = - e^{-xt} \sin^2 x$, etc.

Here, I use some of my commands defined above: I like $\eps = \varepsilon$
better than the default $\epsilon$. A partial derivative (with 2 arguments)
would be obtained as follows. If $f(x,y) = x^2 y^3$, then \premathpar
\begin{displaymath}
  \der{f}{x} = 2 x y^3, \quad \der{f}{y} = 3 x^2 y^2.
\end{displaymath}
\section{Sums and Integrals}
When you say ``capital sigma,'' you probably did not really mean $\Sigma$,
but rather a summation symbol. You would get that as in\premathpar
\begin{displaymath}
  \sum_{i=0}^{\infty} r^i = \frac{1}{1 - r} \quad \mbox{for all $|r| < 1$}.
\end{displaymath}
Finally, we have\premathpar
\begin{displaymath}
  \int_0^1 \sin(2 \pi x) \, dx = 0
\end{displaymath}
and\premathpar
\begin{displaymath}
  \int \! \! \int f(x) g(y) \, dx \, dy = \int f(x) \, dx \,\, \int g(y) dy.
\end{displaymath}
Here, \verb+\,+ gives a small space, while \verb+\!+ forces things closer
together; you have to work on the proper spacing for integrals, as \LaTeX\
does not understand, what is going on.
\clearpage % hack to move section to next page
\section{Matrices in \LaTeX}
A matrix $A \in \mathrm{R}^{m \times n}$ could be defined by\premathpar
\begin{displaymath}
  A = \left( \begin{array}{ccccc}
        11     & 12     & 13     & \cdots & 1n     \\
        21     & 22     & 23     & \cdots & 2n     \\
        \vdots & \vdots & \vdots & \ddots & \vdots \\
        m1     & m2     & m3     & \cdots & mn     \\
      \end{array} \right)
\end{displaymath}
Here, the word \verb+dots+ in the commands stands for an ellipsis
(i.e., three dots) placed horizontally in the centre (\verb+\cdots+),
vertically (\verb+\vdots+), or diagonally (\verb+\ddots+); what is
not mentioned is \verb+\ldots+ for horizontal dots at the baseline.
Use the baseline or central version as appropriate, for instance\premathpar
\begin{eqnarray*}
  a_1, a_2, \ldots, a_n & \mbox{and not} & a_1, a_2, \cdots, a_n, \\
  a_1 + a_2 + \cdots + a_n & \mbox{and not} & a_1 + a_2 + \ldots + a_n, \\
\end{eqnarray*}

Some more comments on the matrix are needed, I suppose:
The \verb+\left(+ and \verb+\right)+ create the variable-sized parentheses
around the actual array of terms. You can also use \verb+\left[+ and
\verb+\right]+, or \verb+\left\{+ and \verb+\right\}+ in other situations.
The actual array arrangement is organised by the \verb+array+ environment;
you need the arguments \verb+ccccc+ to indicate that there are five columns
and you want the entries centered (``c''), other options are left (``l'')
and right (``r''). Notice how \verb+&+ separate columns and \verb+\\+
the rows.

Here is another matrix example.
A matrix multiply used with 3D graphics:
\begin{displaymath}
  \left[ \begin{array}{cccc}
        R_{11} & R_{12} & R_{13} & 0 \\
        R_{21} & R_{22} & R_{23} & 0 \\
        R_{31} & R_{32} & R_{33} & 0 \\
        0      & 0      & 0       & 1
    \end{array} \right]
  \cdot
  \left[ \begin{array}{cccc}
      1 & 0 & 0 & X \\
      0 & 1 & 0 & Y \\
      0 & 0 & 1 & Z \\
      0 & 0 & 0 & 1
  \end{array} \right] 
=  \left[ \begin{array}{cccc}
        R_{11} & R_{12} & R_{13} & T_x \\
        R_{21} & R_{22} & R_{23} & T_y \\
        R_{31} & R_{32} & R_{33} & T_z \\
        0      & 0      & 0       & 1
  \end{array} \right] 
\end{displaymath}


%%% Local Variables: 
%%% mode: latex
%%% TeX-master: "main"
%%% End: 
