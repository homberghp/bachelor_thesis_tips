\renewcommand\TheFile{ch02_motivates.tex}

\begin{savequote}[15cm]
  \vspace{-30mm}
  \raggedleft
  \sffamily
  What moves, motivates
  \qauthor{Ancient wisdom of technical teachers}
\end{savequote}
\chapter{Motivation}
To write a simple document, like a letter, a word processing package 
is just fine.
To automate the creation of a document this is a bit harder. 
Using a word processor to automatically create a consistent document from 
unrelated sources is difficult at best, if doable at all.

However, as long as the only thing that those unrelated sources should
produce are simple ASCII documents, things get much more
doable. It is like comparing HTML to Microsoft Word documents. The power
you have in defining the layout of a document in HTML (combined with
css) with a ``simple'' ASCII document is almost as powerful as what you
can do with Word. But now try writing the same thing with a (self-written) program. You know it is easy in HTML (certainly if you ever
did some programming with e.g. PHP), but producing a Word document with
a program is hard work\footnote{It becomes easier in a modern version of
word processing packages, they tend to use XML as their internal document format}. 

If the document should include complex things like mathematical
formulas that are laid out properly, it becomes difficult in HTML too.

Enter \LaTeX.

\section{Multi-person literal work}
One of the much-overlooked advantages of \LaTeX (and of any multi-file
source code application like Java projects) is that the fact that you
can split up the document into multiple but still coherent parts.
This fact allows you to work on one final document with multiple
persons as team members. This is exactly what the informatics teachers do when writing a report for external evaluation.

This makes \LaTeX\ almost ideal for project work in which several
authors have to contribute to the final product and where source files
are shared using a repository. Even more so in
cases where you want to include (part of) program source code
into the document to explain or show certain implementation
aspects. Such source code is not copied and pasted but instead read
on the fly from the original file(s) during text processing. This
allows you to always have the most up-to-date version.

As this sample is a multipart document, it can be used as a reference
or a start for your document.


\section{Good documentation and online support}
%% this paragraph to show dual use of citations.
Most documentation can be found online. Overleaf has very good documentation. Overleaf has as its main product a build service for \LaTeX\ documents and many students are using and have used it successfully.

The original sources of the \LaTeX\ documentation are Lesly Lamports \LaTeX\ Book \textcite{latexbook}, and expanded on that is the \LaTeX\ Companion \parencite{latexcompanion}.

%%% Local Variables: 
%%% mode: latex
%%% TeX-master: "main.tex"
%%% End: 
