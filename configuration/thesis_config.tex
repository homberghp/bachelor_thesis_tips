%----------------------------------------------------------
% DO NOT TOUCH THIS FILE UNLESS YOU KNOW WHAT YOU ARE DOING
%----------------------------------------------------------
\InputIfFileExists{infopage.txt}{%
  \def\InfoMissingWarning{}
}{%
  \def\InfoMissingWarning{
    
    {\centering \Large \bf The information in this page is not correct, please
      provide the information in the document infopage.txt in the root of this TeX project.}

    You can find an example infopage.text file name infopage.txt-example. One approach is copy and edit the copy.
  }
}

\providecommand\Languages{dutch,ngerman,english}
\documentclass[a4paper,11pt]{report}
% my own Itemize, Enumerate, and Description : a bit less spacy then the default
% my own Itemize, Enumerate, and Description : a bit less spacy then the default
\newenvironment{Itemize} {
	\begin{itemize}{}%
		\setlength\topsep{0ex}%
		\setlength\parskip{0ex}%
		\setlength\partopsep{0em}%
		\setlength\parsep{0em}%
		\setlength\itemsep{0em}%
	}{\end{itemize}
}
\newenvironment{Enumerate}{
	\begin{enumerate}
		\setlength{\itemsep}{1pt}
		\setlength{\parskip}{0pt}
		\setlength{\parsep}{0pt}}
	{\end{enumerate}
}

\newenvironment{Description}{
	\begin{description}
		\setlength{\itemsep}{1pt}
		\setlength{\parskip}{0pt}
		\setlength{\parsep}{0pt}}
	{\end{description}
}
% load early
\usepackage[utf8]{inputenc}
\usepackage[T1]{fontenc}

%% citation and bibliography style.
%% This style depends on 'sortname' and 'shorthand' fields in the bib file.
%% This gives you control on what is shown in the labels.
%% If you ommit the shorthand, the styles fall back to a numeric label

%% biblatex options, defualt to numeric but also support
%% shorthand and sortname keys in bib file bibliography.
\usepackage[
backend=biber,
hyperref=true,
backref=true,
]{biblatex}
\DefineBibliographyStrings{english}{
  backrefpage={cited on p.},
  backrefpages={cited on pp.}
}
% end citation bibliography style.

% support multiple languages.
\usepackage[\Languages]{babel}
% define page layout
\usepackage[a4paper,includemp,pdftex,
      top=20mm,left=20mm,
      total={180mm,260mm},
      includeheadfoot,
      marginparwidth=30mm,
%      showframe % for debugging
      ]{geometry}
      %%
\newcommand\Margin[1]{\marginpar{\sffamily\textbf{#1}}}
      
\usepackage{layout}
      
\usepackage{fancybox}
\usepackage[pdftex]{graphicx}
\usepackage{longtable}
\usepackage{times}
\usepackage{varioref}
% Use multiple references to the same footnote

\usepackage{amsmath}
\usepackage{textcomp}
\usepackage{wrapfig}
\usepackage{varioref}
\usepackage[dvipsnames]{xcolor}
\usepackage[avantgarde]{quotchap}
\usepackage{tikz}
% allow compact lists etc.
\usepackage{enumitem}
\usepackage{%
  array,
  booktabs,
  dcolumn,
  rotating,
  shortvrb,
  units,
  url,
  lastpage,
  longtable,
  lscape,
  multirow,
  amssymb,
  amsmath,
  float,
  chngpage,
  colortbl,
  times,
  csquotes,
}

\usepackage[yyyymmdd]{datetime}
\renewcommand{\dateseparator}{-}
\usepackage{listings}
\lstset{numbers=right} % to get out of the way with document line numbers
\definecolor{listinggray}{gray}{0.2}
\definecolor{lbcolor}{rgb}{0.95,0.98,0.98}

%% settings for listings and code snippets.
\lstset{
  backgroundcolor=\color{lbcolor},
  tabsize=4,
  %	rulecolor=,
  language=java,
  basicstyle=\scriptsize,
  upquote=true,
  aboveskip={1.5\baselineskip},
  columns=fixed,
  showstringspaces=false,
  extendedchars=true,
  breaklines=true,
  prebreak = \raisebox{0ex}[0ex][0ex]{\ensuremath{\hookleftarrow}},
  frame=single,
  showtabs=false,
  showspaces=false,
  showstringspaces=false,
  identifierstyle=\ttfamily\bfseries\color{Black},
  keywordstyle=\color{Violet}\bfseries,%[rgb]{0,0,1},
  commentstyle=\color{Sepia},
  xleftmargin=3.4pt, %% small margins
  xrightmargin=7.4pt, %% margin into marginparsep
  stringstyle=\color[rgb]{0.627,0.126,0.941}
}

%% to show bibtex entries in listintgs examples chapter.
\makeatother
\lstdefinelanguage{BibTeX}
{keywords={%
    @article,@book,@collectedbook,@conference,@electronic,@ieeetranbstctl,%
    @inbook,@incollectedbook,@incollection,@injournal,@inproceedings,%
    @manual,@mastersthesis,@misc,@patent,@periodical,@phdthesis,@preamble,%
    @proceedings,@standard,@string,@techreport,@unpublished%
  },
  comment=[l][\itshape]{@comment},
  sensitive=false,
}

\usepackage[final]{pdfpages}

\usepackage[pagewise,mathlines,displaymath]{lineno}
%\setpagewiselinenumbers
\modulolinenumbers[3]
\renewcommand{\linenumberfont}{\normalfont\tiny\color{gray}}
\usepackage{hyperref}
\hypersetup{
  colorlinks=true,
  colorlinks=true,
  linkcolor=BlueViolet,
  filecolor=DarkOrchid,
  citecolor=PineGreen,
  urlcolor=RoyalPurple,
}
\usepackage[toc,acronym,xindy]{glossaries-extra}
\setabbreviationstyle[acronym]{short-long}
\usepackage{cleveref}[2012/02/15]
\crefformat{footnote}{#2\footnotemark[#1]#3}


\usepackage{lipsum}
\usepackage{metainfo}
\providecommand{\MineOnlyStart}{}
\providecommand{\MineOnlyEnd}{}

\input{configuration/macros}

%\usepackage{natbib}
\usepackage[nottoc]{tocbibind}
\usepackage{titlesec}
\usepackage{fancyhdr}
\addtolength\headwidth\marginparwidth
\addtolength\headwidth\marginparsep
\setlength{\parindent}{0pt}
\setlength{\parskip}{.5\baselineskip}
\usepackage{adjustbox}
\usepackage[toc,page,title,titletoc,header]{appendix}

% chapter titles theme.
\definecolor{gray75}{gray}{0.75}
\newcommand{\hsp}{\hspace{20pt}}
\titleformat{\chapter}[hang]{\Huge\bfseries}{\thechapter\hsp\textcolor{gray75}{|}\hsp}{0pt}{\Huge\bfseries}

% this alters "before" spacing (the second length argument) to 0
\titlespacing*{\chapter}{0pt}{0pt}{40pt}
\titlespacing{\chapter}{0pt}{-25mm}{40pt}

\fancypagestyle{fancy}{%
  \fancyhf{}
  \fancyfoot[C]{\thepage}
}

\fancypagestyle{chapter-header-style}{
  \fancyhead[LE,RO]{\thepage}
  \fancyhead[RE,LO]{\chaptername\ \thechapter\ --\ \leftmark}
  \fancyfoot{}
}

%% in the appendix the page numbers are per appendix chapter
\fancypagestyle{appendix}{
  \fancyhf{}
  \fancyhead[LE,RO]{\thepage}
%  \fancyhead[RE,LO]{\chaptername\ \thechapter\ --\ \leftmark}
  \fancyfoot{}
}

\pagestyle{fancy}

\renewcommand{\chaptermark}[1]{%
  \markboth{#1}{}}

\addto\captionsdutch{%
  \renewcommand\appendixname{Bijlagen}
  \renewcommand\appendixpagename{Bijlage}
  \renewcommand{\refname}{Referenties}
  \renewcommand{\bibname}{Referenties}
}
\addto\captionsgerman{%
  \renewcommand\appendixname{Anhänge}
  \renewcommand\appendixpagename{Anhang}
  \renewcommand{\refname}{Verweise}
  \renewcommand{\bibname}{Verweise}
}
\providecommand\DefInfo[1]{ define \textbackslash{}#1 in file ``./infopage.txt''}
\providecommand\documentname{\DefInfo{documentname}}
\providecommand\studentname{\DefInfo{studentname}}
\providecommand\place{\DefInfo{place}}
\providecommand\snumber{\DefInfo{snummer}}
\providecommand\course{\DefInfo{course}}
\providecommand\period{\DefInfo{period}}
\providecommand\companyname{\DefInfo{CompanyName}}
\providecommand\companyaddress{\DefInfo{CompanyAddress}}
\providecommand\companypostcodecity{\DefInfo{companypostcodecity}}
\providecommand\companycountry{\DefInfo{companycountry}}
\providecommand\companycoach{\DefInfo{companycoach}}
\providecommand\companycoachmail{\DefInfo{companycoachmail}}
\providecommand\universitytutor{\DefInfo{universitytutor}}
\providecommand\universitytutormail{\DefInfo{universitytutormail}}
\providecommand\examiner{\DefInfo{examiner}}
\providecommand\externalexpert{\DefInfo{externalexpert}}
\providecommand\hasnda{\DefInfo{hasnda}}

% This macro '\define' puts the argument in em 
% and in boldface in the margin.
\providecommand{\define}[1]{% 1 argument
  \mbox{}{\textit{#1}}% italics or em
  \marginpar{\raggedright% no adjust
    \bfseries\hspace{0pt}#1}% bold
} % end of macro
\newcommand\Code[1]{\textbf{\texttt{#1}\ }}
\newcolumntype{g}[1]{%
    >{\raggedright\arraybackslash}%
    p{#1}%
}

\newcommand\itemcell[1]{%
  %\mbox{% uncomment for debug 
  \begin{minipage}[t]{\linewidth}%
      \begin{itemize}[leftmargin=3.5ex]%
        #1
      \end{itemize}%
  \end{minipage}\vspace{.4\baselineskip}%
  %}% uncomment
}


\newcommand\headcell[1]{%
  \begin{minipage}[c]{\linewidth}
    \vspace{.2\baselineskip}%
    \raggedright%
    \textbf{#1}
  \end{minipage}\vspace{.4\baselineskip}%
}

\newcommand\firstcell[1]{%
  \cellcolor{Gray}%
  \begin{minipage}[c]{\linewidth}%
    \vspace{.2\baselineskip}%
    \raggedright%
    \textbf{#1}
  \end{minipage}\vspace{.4\baselineskip}%
}

\newcommand\textcell[1]{%
  \fbox{%
    \begin{minipage}[t]{\linewidth}%
      \raggedright%
      #1
  \end{minipage}}\vspace{.4\baselineskip}%
}
\newlength{\BulletSize}
\setlength{\BulletSize}{1.5ex}
\definecolor{Gray}{gray}{0.9} % color used in table borders
\newlength{\TabColumnWidth} % for tables
\newcommand\CheckMarkGreen[1]{\includegraphics[height=#1]{images/Check_Green_White.pdf}\hspace{2pt}}
\newcommand\CMark{\raisebox{-.1\BulletSize}{\CheckMarkGreen{\BulletSize}}}
\newcommand\CrossMarkRed[1]{\includegraphics[height=#1]{images/Cross_red_circle.pdf}\hspace{2pt}}
\newcommand\XMark{\raisebox{-.1\BulletSize}{\CrossMarkRed{\BulletSize}}}
\newcommand\WarningMarkYellow[1]{\includegraphics[height=#1]{images/Alert_Yellow_White.pdf}\hspace{2pt}}
\newcommand\WMark{\raisebox{-.1\BulletSize}{\WarningMarkYellow{\BulletSize}}}
\newcommand\FirstCell[1]{\cellcolor{Gray}\thead{#1}}
\newcommand\URLIcon[1]{\includegraphics[height=#1]{images/URL.pdf}\hspace{2pt}}
\newcommand\UMark{\raisebox{-.1\BulletSize}{\URLIcon{\BulletSize}}}
\newcommand\itemC{\item[\CMark]}
\newcommand\itemX{\item[\XMark]}
\newcommand\itemW{\item[\WMark]}
\newcommand\itemU{\item[\UMark]}
\newcommand\MyURL[2]{\url{#1}{\UMark #1}}
\usepackage[avantgarde]{quotchap}
\renewcommand\chapterheadstartvskip{\vspace*{-5\baselineskip}}

\definecolor{codegray}{gray}{0.9}
\newcommand{\code}[1]{\colorbox{codegray}{\texttt{#1}}}
\usepackage{moreverb}
\IfFileExists{wordcount.txt}{}{
  % not exists
  \immediate\write18{texcount  -inc -sum  -1 chapters/*.tex > wordcount.txt}
}


\InputIfFileExists{IncludeOnly.tex}{}{}
